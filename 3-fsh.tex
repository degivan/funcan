\subsection{Альтернатива Фредгольма-Шаудера}

\begin{definition}[Компактный оператор]
    Оператор $A: X \rightarrow Y$ называется \textit{компактным} если $\forall M$ - ограниченное множество, $A(M)$ - относительный компакт.
\end{definition}

\todo пример оператора фредгольма.

\begin{statement}[Компактность произведения]
    Пусть $A$ - компактный оператор, а $B$ - ограниченный. Тогда $AB$ и $BA$ - компактны.
\end{statement}
 
\begin{proof}
    Достаточно проверить компактность в единичном шаре.
    Пусть $V_1$ - замкнутый шар единичного радиуса, проверим, что $B(A(V_1))$ - относительный компакт.
    Так как $A(V_1)$ - относительный компакт, можем подобрать конечную $\epsilon$-сеть. 
    $\forall \epsilon \exists y_1,  \dots, y_n \forall x \in V_1 \exists j \norm{A(x) - y_j} < \epsilon$. Пусть $z_j = B(y_j)$. 
    $\forall x \in V_1 \norm{B(A(x)) - z_j} = \norm{B(A(x) - y_j)} \leqslant \norm{A(x) - y_j} \cdot \norm{B} \leqslant \epsilon\norm{B}$
\end{proof}

\begin{statement}
    В бесконечномерных пространствах, компактный оператор не может быть непрерывно обратимым.
\end{statement}
\begin{proof}
    Пусть $A$ - компактный, $\exists A^{-1}$ - непрерывный. Тогда $I = A\cdot A^{-1}$ - компактный. Однако, это не так (потому что бесконечномерная сфера - не компакт).
\end{proof}

В классе сепарабельных банаховых пространств важную роль имеют пространства с базисом \textit{Шаудера}.

\begin{definition}[базис Шаудера]
    Пусть $X$ - баназово пространство. $\exists e_1, \dots, e_n, \dots$ - линейно независимые точки. $\forall x \in X$, $x = \sum_{1}^{+\infty} \alpha_j e_j$.
    Тогда ${e_j}$ - называется \textit{базисом Шаудера}.
\end{definition}

\begin{theorem}[О почти конечномерности компактного оператора]
    Пусть $A: X \rightarrow X$ - компактный оператор. $X$ - имеет базис Шаудера. Тогда 
    $\forall \epsilon > 0 \exists B, C: \dim R(B) < +\infty, \norm{C} \leqslant \epsilon, A = B + C$
\end{theorem}

\begin{proof}
    По условию, $x = \sum_{1}^{\infty}\alpha_j e_j$, пусть $S_n(x) = \sum_{1}^{n} \alpha_j e_j$, $R_n(x) = (I - S_n)x$. 
    Проверим, что оператор $S_n$ непрерывен. Для начала, рассмотрим пространство $F = 
    \left\{(\alpha_1, \dots, \alpha_n, \dots) \mid \sum_{1}^{\infty}\alpha_j e_j \text{- сходится в} X \right\}$.
    Введем на этом множестве норму $\norm{\alpha} \defeq \sup_{n} \norm{S_n(x)}$. Очевидно, что это норма, также можно показать, что $F$ - банахово (Пространство $F$ - 
    это по сути то же пространство $X$ только в координатном смысле).
    \todo Рассмотрим оператор $T: F \rightarrow X$, $T(x) = \sum_{0}^{\infty} \alpha_j e_j$ - это линейный оператор.
    Очевидно, $\norm{T(\alpha)} \leqslant \norm{\alpha}$. В силу единственности разложения в ряд и того, как действует $T$, у него существует обратный оператор,
    который, по теореме Банаха о гомеоморфизме, будет ограниченным. $T^{-1}: X \rightarrow F$, $\norm{T^{-1}(x)} \leqslant m\cdot \norm{x}$.
    С другой стороны, $T^{-1}(x) = \alpha$, $\norm{\alpha} \leqslant m \cdot \norm{x} \implies \forall n \norm{S_n(x)} \leqslant \norm{x}$.
    Последовательность операторов $S_n$ - поточетчно сходится, и каждый из них непрерывен, тогда по теореме Банаха-Штейнгауза $\sup_n{\norm{S_n}} = M < \infty$.
    Итак, мы доказали, что $I = S_n + R_n$, где $S_n$ - непрерывный. Очевидно, что $R_n$ - тоже непрерывен.
    Напишем тождество: $A = S_n A + R_n A$. Из них $S_n A$ - конечномерный. Проверим, что $\norm{R_n} < \epsilon$.

    Так как оператор $A$ - компактный, то у множества $A(V_1)$ есть конечная $\epsilon$-сеть. Итак, $\forall \epsilon$ $\exists y_1, \cdots, y_n $ 
    - $\epsilon$-сеть для $A(V_1)$. $\forall x \in V_1 \norm{R_n(Ax)} \leqslant \norm{R_n(Ax - y_j)} +\norm{R_n(y_j)} \leqslant \norm{R_n} \cdot
    \norm{Ax - y_j} \leqslant \epsilon$.
    Первое слагаемое меньше $\epsilon$ за счет $\epsilon$-сети, второе, становится маленьким, за счет увеличения $n$.
\end{proof}

\begin{statement}
    Если $A$  - компактный оператор, то $A^*$ - тоже компактный.
\end{statement}

\begin{proof}
    \todo
\end{proof}

Для функционального анализа фундаментальную роль играют уравнения вида 
\begin{gather}
    y = (\lambda I - A)x
\end{gather}
и, в частности, $y = (I - A)x = Tx$. 


\begin{statement}
    Пусть $A$ - компактный оператор. Тогда $\dim \Ker T < \infty$.
\end{statement}

\begin{proof}
    Пусть $x \in \Ker T \implies x = Ax$. Таким образом, ядро - это множество неподвижных точек оператора $A$. И $\Ker T$ - подпространство $X$.
    Рассмотрим $V_1 \subset \Ker T$ - единичная сфера в этом подпространстве. Тогда, очевидно, $A(V_1) = V_1$. По лемме Рисса о почти перпендикуляре, в бесконечномерном 
    пространстве, единичная сфера - не компакт. А у нас - компакт, значит,  пространство не бесконечномерное.
\end{proof}


\begin{theorem}
    Пусть оператор $A$ - компактный. Тогда $\Cl R(T) = R(T)$ ($R(T)$ - подпространство $X$).
\end{theorem}

\begin{proof}
    Пусть $\Ker T =\mathcal{L}(\phi_1, \cdots, \phi_n)$. Для доказательства замкнутости, нужно показать существование априорной оценки. 
    $y$ можно представить как $y = T\left( x +\sum_{1}^{n}\alpha_k \phi_k \right)$. Чтобы убедиться в существовании оценки, возьмем 
    \begin{gather}
    \widehat{x} = \min_{\alpha_j, j=\overline{1,n}}\norm{x + \sum_{1}^{n}\alpha_k \phi_k}
    \end{gather}
    Покажем, что $\widehat{x}$ всегда существует и $\exists \alpha: \norm{\widehat{x}}\leqslant\alpha\cdot\norm{y}$.
    Пусть $f(\alpha_1, \cdots, \alpha_n) = f(\boldsymbol{\alpha}) = \norm{x + \sum_{1}^{n}\alpha_k \phi_k}$.
    $\abs{f(\boldsymbol{\alpha} + \delta\boldsymbol{\alpha}) - f(\boldsymbol{\alpha})} \leqslant \sum_{1}^{n}\abs{\alpha_k} \cdot\norm{\delta\boldsymbol{\alpha}}
    \leqslant \sqrt{\sum_{1}^{n}\norm{\phi_k}^2}\cdot\sqrt{\sum_{1}^{n}\delta\alpha_k^2}$.
\end{proof}
