\section{Теорема Гильберта-Шмидидта}
\begin{definition}[Гильбертово пространство]
    $H$ - гильбертово пространство над $\mathbb{C}$, если.
    \begin{itemize}
        \item $\scalarp{x}{y} = \overline{\scalarp{y}{x}}$
        \item $\scalarp{x}{\alpha y} = \bar{\alpha}\scalarp{x}{y}$
    \end{itemize}
\end{definition}

\begin{example}[$\mathbb{C}^n$ -- конечномерный случай]
    $z = (z_1, z_2, \dotsc, z_n), z_i \in \mathbb{C} \\
    \scalarp{z}{u} = \sum_{i = 1} ^n z_i \bar{u_i} \\
    \scalarp{x}{x} \geqslant 0, \scalarp{x}{x} = 0 \Leftrightarrow x = 0 \\
    \scalarp{x}{x} = \sum_{i = 1} ^n x_i \cdot \bar{x_i} = \sum_{i = 1} ^n |x_i|^2$
\end{example}

\begin{definition}[Самосопряженный оператор]
    Линейный оператор $A: X \rightarrow X$ называется самосопряженным, 
    если выполняется условие:
    $\scalarp{Ax}{y} = \scalarp{x}{Ay}$.
\end{definition}

\begin{example}
    $A : \mathbb{C}^n \rightarrow \mathbb{C}^n$, 
    $A = \bar{A}^T$ -- эрмитовски симметричная
\end{example}

\begin{statement}
    Если  $\int_a ^b K(k, t) f(t) dt$ - интегральный оператор, 
    то $K(x, y) = \overline{K(y, x)}$
\end{statement}
\begin{proof}
    \todo{\newline Домножить векторы на матрицу поворота}
\end{proof}

\begin{statement}
    $A$ -- самосопряженный оператор, $\lambda = \mu + i \nu$.
    Тогда $\forall x \in H \Rightarrow 
    \norm{Ax - \lambda x} \geqslant |\nu| \cdot \norm{x}$
\end{statement}
\begin{proof}\
    \begin{enumerate}
    	\item Если матрица состоит из вещественных чисел:
            $\norm{Ax - \lambda x}^2 = \scalarp{Ax - \lambda x}{Ax - \lambda x} = \\ 
            = (\underbrace{Ax - \mu x} _{Lx, \mu \in \mathbb{R}} - i \nu x,
            Ax - \mu x - i \nu x) =
            \underbrace{\scalarp{Lx}{Lx}}_{\norm{Lx}^2 = 0} - 
            \underbrace{
                \underbrace{\scalarp{Lx}{i \nu x}} _{- i \nu \scalarp{Lx}{x}} - 
                \underbrace{\scalarp{i \nu x}{Lx}}_{i \nu \scalarp{Lx}{x}}
            }_{= 0} + 
            |\nu|^2 \norm{x}^2 \geqslant |\nu|^2 \norm{x}^2$
        \item Если $Im(a) \neq 0, \\ 
            Ker(A - \lambda i) = \{0\} \Rightarrow$
            Cобственные числа могут быть только вещественными.
            $\Rightarrow |\nu| = |\nu|\norm{x} = 0$
    \end{enumerate}
\end{proof}

\begin{statement}
    Пусть $A$ - линейный оператор в $H$, тогда $Cl\, R(A) = (Ker A^*)^{\perp}$
\end{statement}
\begin{proof}
    \begin{nb}
        В общем случае уже доказывали, сейчас выведем независимо.


        \todo{Написать, где доказательство.}
    \end{nb}
    \begin{enumerate}
        \item Рассмотрим $y \in R(A): \\
            \forall z \in Ker A^* \Leftrightarrow
            A^*z = 0 \Rightarrow
            \scalarp{y}{z} = \scalarp{Ax}{z} = \scalarp{x, A^*z} = \scalarp{x}{0} = 0 \\
            \text{Taким образом, } \forall y \in R(A) \Rightarrow
            y \perp A^* \\
            y \in Cl\, R(A), y_n \in R(A), y_n \rightarrow y \Rightarrow
            0 = \scalarp{y_n}{z} \rightarrow \scalarp{y}{z} \Rightarrow
            \scalarp{y}{z} = 0 \\
            $Итак, $ Cl\, R(A) \subset (Ker A^*)^{\perp}.$
        \item Проверим обратное включение
            $y \notin Cl\, R(A) \Rightarrow
            y \notin (Ker A^*)^{\perp}. \\
            Cl\, R(A) = H_1$ -- подпространство $H,
            H = H_1 \oplus H_1 ^\perp. \\
            $Рассмотрим $ y \notin H_1, 
            \exists y_1 \in H_1, y_2 \in H_1 ^\perp, y_2 \neq 0 :\,
            y = y_1 + y_2 \\
            y_2 \in Ker A^* \Leftrightarrow
            A^*y_2 = 0 \Leftrightarrow
            \scalarp{A^*y_2}{A^*y_2} = 0 \Leftrightarrow
            (\underbrace{A(A^*y_2)}_{\in H_1},
            \underbrace{y_2}_{\in H_1 ^\perp}) = 0 \\
            \scalarp{y}{y_2} = \norm{y_2}^2 \neq 0 \Rightarrow
            y \notin (KerA^*)^\perp$
    \end{enumerate}
\end{proof}
\begin{corollary}
    $H = Cl\, R(A) \oplus Ker A^*$
\end{corollary}

\begin{theorem}
    Пусть оператор $A$ самосопряженный,тогда $\sigma(A) \subset R$.
\end{theorem}
\begin{proof}
    $\lambda : Im \lambda \neq 0,\,
    R(A - \lambda x)$ -- замкнуто $\\
    (A - \lambda I)^* = A - \bar{\lambda} I \\
    \norm{(A -\bar{\lambda} I)x} \geqslant
    \underbrace{|Im \lambda|}_{\neq 0} \cdot \norm{x} \\
    Ker(A - \lambda I)^* = \{0\}, H = R(A - \lambda I) \\
    По теореме Банаха о гомоморфизме: \exists (A - \lambda I)^{-1}
    \Rightarrow \lambda \in \dot{\rho}(A)$
\end{proof}

\begin{theorem}
    $A$ -- самосопряженный и ограниченный в $H$.
    \begin{enumerate}
        \item
            $\lambda \in \rho \Leftrightarrow 
            \exists m > 0: \forall x \in H \quad
            \norm{(A - \lambda I)x)} \geqslant m \norm{x}$
        \item
            $\lambda \in \sigma(A) \Leftrightarrow
            \exists x_m: \norm{x_m} = 1 \quad
            \norm{(A - \lambda I)x_m} \rightarrow 0$
    \end{enumerate}
\end{theorem}
\begin{proof}\
    \begin{enumerate}
        \item $\Rightarrow: \\
            \lambda \in \rho (A) \Leftrightarrow \exists (A - \lambda I)^{-1}
            \norm{(A - \lambda I)^{-1}} \cdot M \\
            \norm{(A - \lambda I)^{-1}y} \leqslant M \norm{y} \\
            y = (A - \lambda I)x \\
            \frac{m}{M} \norm{x} \leqslant \norm{(A -\lambda I)x} \\
            \\ \Leftarrow: \\$
            Если неравенство выполнено, то $R(A - \lambda I)$ - замкнут.$\\
            Im(\lambda) \neq 0,\, \lambda = \bar{\lambda},
            (A - \lambda I)^* = A - \lambda I,\,
            Ker(A - \lambda I) = \{0\} \\
            H = R(A - \lambda I)\,$
            Значит, он сюрьективный.
        \item
            Математическое отрицание пункта 1,
            но $Ax_n \approx \lambda x_n. \\$
            То есть, у самосопряженного оператора спектр состоит из почти собственных значений.
            \begin{gather}
                \scalarp{Ax}{x} = \overline{\scalarp{x}{Ax}} =
                \overline {\scalarp{Ax}{x}} \Rightarrow
                \forall x \in H \quad \scalarp{Ax}{x} \in \mathbb{R}
            \end{gather}
            Это называется кр. ф-ой самосопряженного оператора.

            \todo{ Расшифровать сокращения в предыдущей строчке.}
    \end{enumerate}
\end{proof}

\begin{definition}
    Нижняя и верхняя граница оператора $A \\
    m_- = \inf_{\norm{x} = 1} \scalarp{Ax}{x} \quad
    m_+ = \sup_{\norm{x} = 1} \scalarp{Ax}{x}$
\end{definition}

\begin{statement}
    $|\scalarp{Ax}{x}| \leqslant \norm{Ax} \cdot \norm{x} \leqslant
    \norm{A} \cdot \norm{x}^2 = \norm{A} \\$
    То есть $|m_-|, |m_+| \leqslant$
\end{statement}
\begin{proof}\
    $m_- \leqslant \scalarp{Ax}{x} \leqslant m_+ \quad \norm{x} = 1 \\
    m_- \underbrace{\norm{x}^2}_{\scalarp{x}{x}}
    \leqslant \scalarp{Ax}{x} \leqslant m_+ \norm{x}^2 \Rightarrow
    0 \leqslant \scalarp{A - m_- I)x}{x}$
\end{proof}

\begin{theorem}\
    \begin{enumerate}
    	\item $\sigma(A) \subset [m_-, m_+]$
    	\item $m_-, m_+ \in \sigma(A)$
    \end{enumerate}
\end{theorem}
\begin{proof}\
    \begin{enumerate}
    	\item
    	    Так как числа с ненулевой мнимой частью резольв. множество,
    	    рассмотрим только вещественные.
    \end{enumerate}
    \todo{ Расшифровать сокращение "резольв".}
\end{proof}
\pagebreak