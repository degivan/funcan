\section{Элементы спектральной теории линейных операторов}

\subsection{Определение спектра и резольвенты оператора}

\begin{definition}[регулярная точка]
    Число $\lambda \in \mathbb{C}$, называется \emph{регулярной точкой}
    для оператора $A$, если оператор $\lambda I - A$ "--- непрерывно обратим.
\end{definition}

\begin{definition}[резольвента]
    Множество всех регулярных точек называется \emph{резольвентой} (обозначается $\rho(A)$)
    оператора $A$.
\end{definition}

\begin{definition}[резольвентный оператор]
    Оператор $R_\lambda(A) = \left( \lambda I - A\right)^{-1}$ называется
    \emph{резольвентным оператором}.
\end{definition}

\begin{definition}[спектр]
    Множество $\sigma(A) = \mathbb{C} \setminus \rho(A)$ называется
    \emph{спектром} оператора $A$.
\end{definition}

Рассмотрим $\lambda \in \sigma(A)$. Может быть два случая:
\begin{enumerate}
    \item
    $\Ker (\lambda I - A) \neq \left\{ \bf{0}\right\}$. Это значит, что оператор
    $\lambda I - A$ имеет нетривиальное собственное подпространство, в котором (по определению)
    выполняется $Ax=\lambda x$, $x \neq \bf{0}$, для некоторых $x$
    (то, что часто называется собственными числами и векторами).
    \item
    $\Ker (\lambda I - A) = \left\{ \bf{0}\right\}$.
    Здесь необходимо рассмотреть два подслучая:
    \begin{enumerate}
        \item
        $\dim X < +\infty$. В конечномерном случае из сюрьективности следует
        биективность, поэтому обратный оператор всегда существует.
        А спектр будет состоять из собственных значений.
        \item
        $\dim X = +\infty$. В этом случае может отсутствовать непрерывная обратимость.
        Если при этом $\Cl R(\lambda I - A) = X$, то говорят, что $\lambda$
        принадлежит непрерывной части спектра. Иначе говорят, что
        $\lambda$ принадлежит остаточной части спектра.
        (те $\lambda$ для которых ядро нетривиально называют дискретной частью спектра).
    \end{enumerate}
\end{enumerate}

\begin{statement}
    Резольвентное множество является открытым в $\mathbb{C}$.
\end{statement}
\begin{proof}
    Пусть $\lambda_0 \in \rho(A)$, тогда $\lambda_0 I - A$ - непрерывно обратим.
    Напишем тождество: $\lambda I - A = (\lambda - \lambda_0)I + \lambda_0I - A$.
    $I = (\lambda_0I - A) R_{\lambda_0}(A)$.
    Отсюда, $\lambda I - A = (\lambda_0I - A) \cdot
    (R_{\lambda_0} (\lambda - \lambda_0) - I)$.
    Заметим, что если $\norm{R_{\lambda_0}} \cdot
    \abs{\lambda - \lambda_0} < 1$,
    то по теореме Банаха о
    непрерывной обратимости оператора $I - C$,
    оператор $R_{\lambda_0} (\lambda - \lambda_0) - I$ - непрерывно обратим.
    Получается, что если $\lambda : \abs{\lambda - \lambda_0} <
    \frac{1}{\norm{R_{\lambda_0}}}$, то $\lambda \in \rho(A)$.
\end{proof}

\begin{corollary}
    Спектр - замкнутое множество.
\end{corollary}

\begin{theorem}
    Пусть $A$ - ограничен, тогда $\sigma(A) \neq \varnothing$
\end{theorem}
\begin{definition}[Спектральный радиус оператора]
    \begin{gather*}
        r_{\gamma}(A) \defeq \sup_{\lambda \in \sigma(A)} \abs{\lambda}
    \end{gather*}
\end{definition}
\begin{statement}
    $\exists \lim \sqrt[n]{\norm{A^n}} = r_{\gamma}(A)$
\end{statement}
\begin{proof}
    Очевидно, что всегда существует $\inf_{n}\sqrt[n]{\norm{A^n}} = r$.
    Итак, $\forall \epsilon > 0 \exists n_0: \sqrt[n_0]{\norm{A^{n_0}}} < r + \epsilon$.
    Рассмотрим теперь $n > n_0$. $n = m_n \cdot n_0 + d_n$. Тогда
    \begin{gather*}
        \norm{A^{n}}^{\frac{1}{n}}
        = \norm{A^{m_n \cdot n_0} \cdot A^{d_n}}^{\frac{1}{n}} \leqslant \norm{A^{n_0}}^{\frac{m_n}{n}} \cdot
        \norm{A}^{\frac{d_n}{n}}\text{, где } \norm{A}^{\frac{d_n}{n}} \rightarrow 0. \\
        \norm{A^{n_0}}^{\frac{m_n}{n}} < \left( \norm{A^{n_0}}^{\frac{1}{n_0}}\right)^{\frac{n_0 \cdot m_n}{n}}
        < \left( r + \epsilon\right)^{\frac{m_n \cdot n_0}{n}}
    \end{gather*}
    Следовательно, $r \leqslant \norm{A^n}^{\frac{1}{n}} \leqslant
    (1 + \alpha_n) \cdot \left( r + \epsilon\right)^{1 - \frac{d_n}{n}} =
    (1 - \alpha_n) \cdot (r - \epsilon)^{\frac{d_n}{n}} \cdot (r + \epsilon) =
    (1 + \gamma_n) \cdot (r + \epsilon)$.
\end{proof}
