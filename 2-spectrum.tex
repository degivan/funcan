\section{Элементы спектральной теории линейных операторов}

\subsection{Определение спектра и резольвенты оператора}

\begin{definition}[регулярная точка]
    Число $\lambda \in \mathbb{C}$, называется \textit{регулярной точкой}
    для оператора $A$, если оператор $\lambda I - A$ "--- непрерывно обратим.
\end{definition}

\begin{definition}[резольвента]
    Множество всех регулярных точек называется \textit{резольвентой} (обозначается $\rho(A)$)
    оператора $A$.
\end{definition}

\begin{definition}[резольвентный оператор]
    Оператор $R_\lambda(A) = \left( \lambda I - A\right)^{-1}$ называется 
    \textit{резольвентным оператором}.
\end{definition}

\begin{definition}[спектр]
    Множество $\sigma(A) = \mathbb{C} \setminus \rho(A)$ называется 
    \textit{спектром} оператора $A$.
\end{definition}

Рассмотрим $\lambda \in \sigma(A)$. Может быть два случая:
\begin{enumerate}
    \item
    $\Ker (\lambda I - A) \neq \left\{ \boldsymbol{0}\right\}$. Это значит, что оператор 
    $\lambda I - A$ имеет нетривиальное собственное подпространство, в котором (по определению)
    выполняется $Ax=\lambda x$, $x \neq \boldsymbol{0}$, для некоторых $x$
    (то, что часто называется собственными числами и векторами).
    \item
    $\Ker (\lambda I - A) = \left\{ \boldsymbol{0}\right\}$. 
    Здесь необходимо рассмотреть два подслучая:
    \begin{enumerate}
        \item
        $\dim X < +\infty$. В конечномерном случае из сюрьективности следует
        биективность, поэтому обратный оператор всегда существует.
        А спектр будет состоять из собственных значений.
        \item
        $\dim X = +\infty$. В этом случае может отсутствовать непрерывная обратимость.
        Если при этом $\Cl R(\lambda I - A) = X$, то говорят, что $\lambda$ 
        принадлежит непрерывной части спектра. Иначе говорят, что
        $\lambda$ принадлежит остаточной части спектра.
        (те $\lambda$ для которых ядро нетривиально называют дискретной частью спектра).
    \end{enumerate}
\end{enumerate}

\begin{statement}
    Резольвентное множество является открытым в $\mathbb{C}$.
\end{statement}
\begin{proof}
    Пусть $\lambda_0 \in \rho(A)$, тогда $\lambda_0 I - A$ - непрерывно обратим.
    Напишем тождество: $\lambda I - A = (\lambda - \lambda_0)I + \lambda_0I - A$.
    $I = (\lambda_0I - A) R_{\lambda_0}(A)$.
    Отсюда, $\lambda I - A = (\lambda_0I - A) \cdot 
    (R_{\lambda_0} (\lambda - \lambda_0) - I)$.
    Заметим, что если $\norm{R_{\lambda_0}} \cdot
    \abs{\lambda - \lambda_0} < 1$,
    то по теореме Банаха о 
    непрерывной обратимости оператора $I - C$, 
    оператор $R_{\lambda_0} (\lambda - \lambda_0) - I$ - непрерывно обратим.
    Получается, что если $\lambda : \abs{\lambda - \lambda_0} < 
    \frac{1}{\norm{R_{\lambda_0}}}$, то $\lambda \in \rho(A)$. 
\end{proof}

\begin{corollary}
    Спектр - замкнутое множество.
\end{corollary}

\begin{theorem}
    Пусть $A$ - ограничен, тогда $\sigma(A) \neq \varnothing$
\end{theorem}
\begin{definition}[Спектральный радиус оператора]
    \begin{gather*}
        r_{\gamma}(A) \defeq \sup_{\lambda \in \sigma(A)} \abs{\lambda}
    \end{gather*}
\end{definition}
\begin{statement}
    $\exists \lim \sqrt[n]{\norm{A^n}} = r_{\gamma}(A)$
\end{statement}
\begin{proof}
    Очевидно, что всегда существует $\inf_{n}\sqrt[n]{\norm{A^n}} = r$. 
    Итак, $\forall \Epsilon > 0 \exists n_0: \sqrt[n_0]{\norm{A^{n_0}}} < r + \Epsilon$.
    Рассмотрим теперь $n > n_0$. $n = m_n \cdot n_0 + d_n$. Тогда 
    \begin{gather*}
        \norm{A^{n}}^{\frac{1}{n}} 
        = \norm{A^{m_n \cdot n_0} \cdot A^{d_n}}^{\frac{1}{n}} \leqslant \norm{A^{n_0}}^{\frac{m_n}{n}} \cdot 
        \norm{A}^{\frac{d_n}{n}}\text{, где } \norm{A}^{\frac{d_n}{n}} \rightarrow 0. \\
        \norm{A^{n_0}}^{\frac{m_n}{n}} < \left( \norm{A^{n_0}}^{\frac{1}{n_0}}\right)^{\frac{n_0 \cdot m_n}{n}} 
        < \left( r + \Epsilon\right)^{\frac{m_n \cdot n_0}{n}}
    \end{gather*}
    Следовательно, $r \leqslant \norm{A^n}^{\frac{1}{n}} \leqslant 
    (1 + \alpha_n) \cdot \left( r + \Epsilon\right)^{1 - \frac{d_n}{n}} = 
    (1 - \alpha_n) \cdot (r - \Epsilon)^{\frac{d_n}{n}} \cdot (r + \Epsilon) = 
    (1 + \gamma_n) \cdot (r + \Epsilon)$
\end{proof}

\begin{example}
    Пространство $C[0, 1]$. Оператор $A(f, t) = t \cdot f(t)$. Очевидно, что 
    $\norm{A(f)} \leqslant \norm{f}$. Пусть $\lambda I - A = A_\lambda$.
    $A_\lambda(f, t) = (\lambda - t) \cdot  f(t) = g(t) \implies 
    f(t) = \frac{g(t)}{\lambda - t}$. При каких $\lambda$, $A_\lambda$ непрерывно обратим?
    Очевидно, при $\lambda \notin [0, 1]$. Это значит, что если $\lambda \in [0, 1]  \implies \lambda \in \rho(A)$.
    Поэтому $\sigma(A) = [0, 1]$.
\end{example}

\begin{example}
    Пространство $C[0, 1]$. $A(f, x) = \int_{0}^{x}f(t)dt$, $x \in [0, 1]$.
    Вычислим его спектральный радиус.
    
    \begin{gather*}
        A(f, x) = \int_{0}^{x}f(t)dt \\
        A^2(f, x) = \int_{0}^{x} \left( \int_{0}^{x_1} f(t) dt\right)dx_1 \\
        A^n(f, x) = \int_{0}^{x} dx_1 \int_{0}^{x_1} dx_2 \cdots \int_{0}^{x_{n + 1}} f(t)dt \leqslant 
        \frac{\norm{f}}{n!} \\
        \norm{f} \leqslant 1 \implies \norm{A^n} \leqslant \frac{1}{n!} \implies r_n \leqslant \frac{1}{\sqrt[n]{n!}} \rightarrow 0 
        \implies r_{\sigma}(A) = 0
        \end{gather*}
\end{example}


\begin{theorem}[Об отображении спектра полиномами]
$\sigma(P(A)) = P(\sigma(A))$
\end{theorem}

\begin{lemma}
    
$P(A)$ - непрерывно обратим $\Leftrightarrow  0 \notin P(\sigma(A))$
\end{lemma}
\begin{proof}
    \begin{enumerate}
    \item Необходимость.

    Сначала проверим следующий факт. Если два оператора коммутируют и их произведение непрерывно обратимо, то и каждый из них непрерывно обратим.
    $T = A \cdot B = B \cdot A$, $\exists T^{-1}$ - непрерывно обратимый.
    $I = T^{-1} T = T^{-1}\cdot \left( A  B\right) = \left( T^{-1} A \right) \cdot B = B^{-1} B$. Для $A$ аналогично.
    В общем случае, $p(t)$ имеет вид $p(t) = a (t - \lambda_1)^{m_1} \cdot (t - \lambda_2)^{m_2} \cdots (t - \lambda_n)^{m_n}$, 
    а $p(A) = a (A - \lambda_1 I)^{m_1} \cdot (A - \lambda_2 I)^{m_2} \cdots (A - \lambda_n I)^{m_{n}}$. Так как $p(A)$ - непрерывно обратим, то, по доказанному,  
    каждый из множителей непрерывно обратим. Это значит, что каждое из $\lambda_j \in \rho(A)$. Поэтому, если $0 \in p(\sigma(A))$, 
    то одно из $\lambda_j$ является корнем уравнения $p(t) = 0$. То есть, при каком-то $\lambda_j$, $p(\lambda_j) = 0$, это значит что 
    $\lambda_j \in \sigma(A)$. Но $\lambda_j \in \rho(A)$. Противоречие.
    \item Достаточность. Тут все аналогично необходимости, только в другую сторону.
    

    \end{enumerate}
\end{proof}
\begin{proof}[Доказательство теоремы]
    Рассмотрим полином $p_1(t) = p(t) - \lambda$. Тогда, по лемме, $p_1(A)$ - непрерывно обратим тогда и только тогда когда $0 \notin p_1(\sigma(A))$, 
    что эквивалентно $p(A) - \lambda I$ - непрерывно обратим, тогда и только тогда когда $\lambda \notin p(\sigma(A))$. Но $\exists(p(A) - \lambda I)^{-1} \Leftrightarrow 
    \lambda \notin \sigma(p(A))$
\end{proof}
