\documentclass{article}

\usepackage{enumitem,linegoal}
\usepackage{xfrac}
\usepackage{faktor}
\usepackage{stmaryrd}
\usepackage{calc}
\usepackage{wrapfig}
\usepackage{blindtext}
\usepackage{tikz}
\usepackage{yhmath}
\usepackage{amssymb}
\usepackage{amsthm}
\usepackage{amsmath}
\usepackage{mathtext}
\usepackage[T1,T2A]{fontenc}
\usepackage[utf8]{inputenc}
\usepackage[russian]{babel}
\usepackage[left=2cm,right=2cm,top=2cm,bottom=2cm,bindingoffset=0cm]{geometry}
\usepackage[mathscr]{euscript}
\usepackage{microtype}
\usepackage{enumitem}
\usepackage{bm}
\usepackage{listings}
\usepackage{cancel}
\usepackage{proof}
\usepackage{epigraph}
\usepackage{titlesec}
\usepackage{mathtools}
\usepackage[hidelinks]{hyperref}

\selectlanguage{russian}

\hypersetup{%
    colorlinks=true,
    linkcolor=blue
}

\DeclareMathOperator{\grad}{grad}
\DeclareMathOperator{\Div}{div}
\DeclareMathOperator{\rot}{rot}
\DeclareMathOperator{\im}{Im}
\DeclareMathOperator{\Ker}{Ker}
\DeclareMathOperator{\codim}{codim}
\DeclareMathOperator{\Cl}{Cl}
\DeclareMathOperator{\Int}{Int}

\title{%
    \sc{Функциональный анализ, 6 семестр} \\
    \large Конспект лекций Додонова Н. Ю.}
\author{\url{https://github.com/artemZholus/funcan}}
\date{}

\begin{document}

\theoremstyle{definition}
\newtheorem*{definition}{Определение}
\theoremstyle{plain}
\newtheorem{theorem}{Теорема}[section]
\newtheorem{axiom}{Аксиома}
\newtheorem{lemma}[theorem]{Лемма}
\newtheorem{statement}[theorem]{Утверждение}
\newtheorem{nb}[theorem]{N. B.}
\newtheorem{corollary}[theorem]{Следствие}
\theoremstyle{remark}
\newtheorem*{example}{Пример}
\newtheorem{property}[theorem]{Свойство}

\newcommand{\todo}{\textsc{\textbf{TODO}}}
\newcommand{\abs}[1]{\left|#1\right|}
\newcommand{\eqcls}[1]{\left[#1 \right]}
\newcommand{\norm}[1]{\left\|#1\right\|}
\newcommand{\bigslant}[2]{{\raisebox{.2em}{$#1$}\left/\raisebox{-.2em}{$#2$}\right.}}
\newcommand{\normp}[1]{\norm{#1}_p}
\newcommand\restr[2]{{\left.\kern-\nulldelimiterspace #1 \vphantom{\big|} \right|_{#2} }}
\newcommand{\normpp}[2]{\norm{#1}_{#2}}
\newcommand{\defeq}{\mathrel{\stackrel{\makebox[0pt]{\mbox{\normalfont\tiny def}}}{=}}}
\makeatletter
\newcommand*{\rom}[1]{\expandafter\@slowromancap\romannumeral #1@}
\makeatother
\newcommand{\pdiff}[2]{\frac{\partial #1}{\partial #2}}
\newcommand{\feps}{\forall\varepsilon}
\newcommand{\scalarp}[2]{\left\langle #1 , #2\right\rangle}
\newcommand{\set}[1]{\left\{#1\right\}}
\renewcommand{\phi}{\varphi}
\renewcommand{\epsilon}{\varepsilon}

\maketitle
\tableofcontents
\newpage

\section{Линейные операторы в банаховых пространствах}
\subsection{Сопряженный оператор}

Здесь и далее, если не оговорено иного, считаем, что мы находимся в $В$-пространствах.

\begin{definition}[сопряженное пространство]
    $X^* = \left\{ f : X \xrightarrow[\text{непр.}]{\text{лин.}} \mathbb{R} \right\}$ "--- пространство сопряженное к $X$.
\end{definition}
Заметим, что это пространство линейных функционалов, а значит, мы можем ввести в нем норму как норму линейного функционала.
\begin{gather}
    \norm{f} = \sup_{\mathclap{\norm{x} \leqslant 1}} \abs{f(x)}  \\
\end{gather}
По свойствам числовой оси получаем, что $X^*$ всегда банахово (независимо от $X$).

Рассмотрим теперь $A \in \mathcal{L}(X, Y)$. Пусть $f(x) = \phi(Ax)$, где $\phi \in Y^*$.
\begin{definition}
    Сопряженный оператор к $A$ имеет вид $A^*(\phi) = \phi \circ A$.
\end{definition}
\begin{statement}
    Если $A$ нерперывный, то $A^*$ тоже непрерывный.
\end{statement}
\begin{proof}
    Пусть $A$ непрерывен, тогда он ограничен. Тогда справедливо
    \begin{gather}
        \norm{A^*(\phi)} \leqslant \norm{\phi} \cdot \norm{A}\text{.} \label{1.2}
    \end{gather}
    Переходя к $\sup$ по $\phi$ получаем непрерывность $A^*$.
\end{proof}
\begin{theorem} \label{th1.2}
    $\norm{A^*} = \norm{A}$
\end{theorem}
\begin{proof}
    Мы доказали неравенство в одну сторону (неравенство \ref{1.2}). Докажем в другую.
    По определению $\sup$: $\forall \Epsilon > 0, \exists x_{\Epsilon} : \norm{x_{\Epsilon}} = 1 \implies \norm{A} - \Epsilon < \norm{Ax_{\Epsilon}}$.
    Пусть $Z = \mathcal{L}(Ax_{\Epsilon})$. Рассмотрим $f : Z \rightarrow \mathbb{R}$, $f(z) = \alpha\norm{Ax_{\Epsilon}}$.
    Очевидно, что $f \in Y^*$.
    Поэтому, по теореме Хана-Банаха распространим $f$ на все $Y$, и назовем ее $\phi_{\Epsilon}$.
    Тогда, по свойствам $f$, $\norm{\phi_{\Epsilon}} = 1$, $\phi_{\Epsilon}(Ax_{\Epsilon}) = \norm{Ax_{\Epsilon}}$.
    Слeдовательно, $\norm{A} - \Epsilon < \phi_{\Epsilon}(Ax_{\Epsilon}) = A^*(\phi_{\Epsilon}, x_{\Epsilon})$.
    Тогда $\norm{A} - \Epsilon < \norm{A^*} \cdot \norm{\phi_{\Epsilon}} \cdot \norm{x_{\Epsilon}} = \norm{A^*}$.
    Переходя к $\sup$ по $\Epsilon$ получаем нужное неравенство.
\end{proof}

Пример: \todo

\begin{theorem}[теорема Рисса]
    Пусть $H$ "--- гильбертово пространство.
    Тогда $\forall f \in H^*$, $f$ можно представить как $f(x) = \scalarp{x}{y}$, где $y \in H$, $\norm{f} = \norm{y}$.
\end{theorem}
\begin{proof}
    Докажем в 3 этапа.
    \begin{enumerate}
        \item Построим соотвестсвующий функционал по данному $y$. \\
            Пусть $g(x) = \scalarp{x}{y}$. Очевидно, что это линейный функционал.
            По неравенству Шварца $\abs{g(x)} \leqslant \norm{y} \norm{x} \implies \norm{g} \leqslant \norm{y}$.
            Это значит, что $g$ ограничен.
            Возьмем $x = \frac{y}{\norm{y}}$.
            \[
                g\mathopen{}\left(\frac{y}{\norm{y}}\mathclose{}\right) = \scalarp{\frac{y}{\norm{y}}}{y}
                    = \frac{1}{\norm{y}}\scalarp{y}{y} = \norm{y}
            \]
            Сопоставляя это с тем, что $\norm{g} \leqslant \norm{y}$, получаем, что $\norm{g} = \norm{y}$.
        \item Докажем, что этому функционалу соответствует только один $y$. \\
            Пусть для какого-то $\widetilde{y}$ справедливо $g(x) = \scalarp{x}{\widetilde{y}}$.
            Тогда $0 = \scalarp{x}{y} - \scalarp{x}{\widetilde{y}} = \scalarp{x}{y - \widetilde{y}}$.
            Пусть $x = y - \widetilde{y}$, тогда
            $\scalarp{y - \widetilde{y}}{y - \widetilde{y}} = 0 \implies y = \widetilde{y}$
        \item Найдем $y$ для данного функционала $f$. \\
            Рассмотрим произвольный функционал $f \in H^*$.
            Как известно, $\Ker f$ "--- гиперплоскость, т.е.
            $\codim H_1 = \dim H_2 = 1$, где $H_1 = \Ker f$, $H_2 = H_1^\bot$, и $H = H_1 \oplus H_2$.
            Это по определению значит, что $x$ единственным образом представим как $x = x_1 + x_2$, где $x_1 \in H_1$, $x_2 \in H_2$.
            Поэтому, $f(x) = f(x_1) + f(x_2) = f(x_2) = f(\alpha e) = \alpha \cdot f(e)$,
            так как $x_1 \in \Ker f$, а $e$ "--- базисный вектор из $H_2$.
            Итак, $\alpha \cdot f(x) = \scalarp{x}{y} \Leftrightarrow f(e) = \scalarp{e}{y}$.
            Очевидно, $y$ можно брать из $H_2$, так как если у него будет компонента из $\Ker f$, то она будет ортогональна $e$.
            Поэтому, считаем, что $y = \beta e$.
            Получаем $f(e) = \scalarp{e}{\beta e} = \beta \cdot \norm{e}^2$.
            Положим $\beta = \frac{f(e)}{\norm{e}^2}$, тогда $y = \frac{f(e)}{\norm{e}^2}e$.
    \end{enumerate}
\end{proof}


Пример: \todo

Пусть $H = L_2(E)$, $\varphi \in L_2^*(E)$.
Тогда $\varphi(f) = \int_{E} g \cdot f d\mu$.
Согласно теореме Рисса, возвращаясь к сопряженному оператору, мы видим следующее.
$A^*(\varphi, x) = \varphi(Ax) = \scalarp{Ax}{y} = \scalarp{x}{z}$ - последнее равенство по теореме Рисса.
Причем $y$ и $z$ выбираются единственным образом, и $z = A^*(y)$.
В гильбертовом пространстве это может служить определением сопряженного оператора:
\begin{definition}[Сопряженный оператор в гильбертовом пространстве]
    Пусть $x, y \in H$. Пусть $A: H \rightarrow H$. Тогда $A^*$ - такой, что $\scalarp{Ax}{y} = \scalarp{x}{A^*y}$.
\end{definition}

\subsection{Ортогональное дополнение в банаховых пространствах}


\begin{definition}[ортогональное дополнение в $B$-пространстве]
    Пусть $S \subset X$.

    Тогда $S^\bot = \left\{ f \mid f \in X^*, \forall x \in S \implies f(x) = 0 \right\}$.
\end{definition}

\begin{definition}[ортогональное дополнение в сопряженном пространстве]
    Пусть $S \subset X^*$.

     Тогда $S^\bot = \left\{ x \mid x \in X, \forall f \in S \implies f(x) = 0 \right\}$.
\end{definition}
Заметим, что независимо от $S$, $S^\bot$ замкнуто в силу непрерывности $f(x)$

\begin{statement} \ 
    \begin{enumerate}
        \item $X^\bot = \left\{ 0\right\}$;
        \item $X^{*\bot} = \left\{ 0\right\}$.
    \end{enumerate}
\end{statement}
\begin{proof}\ 
    \begin{enumerate}
        \item $f \in X^\bot$. Если $\forall x \in X, f(x) = 0 $, то $f \equiv 0$.
        \item Рассмотрим $\forall f \in X^*$. Очевидно, $f(0) = 0$, а это значит, что $0 \in X^{*\bot}$.
            Предположим, что $\exists x_0 \neq 0 : x_0 \in X^{*\bot}$.
            По теореме \ref{th1.2}:
            \[
                \exists f \in X^* : f(x_0) = \norm{x_0} \neq 0 \implies x_0 \not\in X^{*\bot}\text{.} \qedhere
            \]
    \end{enumerate}
\end{proof}

\begin{definition}[множество значений оператора]
    $R(A) \defeq \left\{ Ax \mid x \in X\right\}$.
\end{definition}

\begin{theorem}
    $\Cl R(A) = \left( \Ker A^*\right)^\bot$.
\end{theorem}
\begin{proof}\ 
    \begin{enumerate}
        \item
            Пусть $y \in R(A)$, это значит, что $y = Ax$ для некоторого $x$.
            Рассмотрим $\phi \in \Ker A^*$. По определению, $A^*\phi = 0$, это значит,
            что $\forall x \in X \implies \phi(Ax) = \phi(y) = 0$.
            Следовательно, $y \in \left( \Ker A^*\right)^\bot$
        \item
            Пусть теперь $y \in \Cl R(A) \implies \exists y_n : y_n \rightarrow y$.
            По предыдущему пункту, $y_n \in \left( \Ker A^*\right)^\bot$.
            $\forall \phi \in \Ker A^* \implies \phi(y_n) = 0$, при этом, $\phi$ непрерывен.
            $\phi(y_n) \rightarrow \phi(y) = 0 \implies y \in \left( \Ker A^*\right)^\bot$.
        \item
            Осталось проверить, что $\left( \Ker A^*\right)^\bot \subset \Cl R(A)$.
            Вместо этого, мы проверим эквивалентный факт:
            $y \not\in \Cl R(a) \implies y \not\in \left( \Ker A^*\right)^\bot$.
            Итак, пусть $L = \Cl R(A)$. Очевидно, это линейное подпространство в $Y$.
            Пусть $\widehat{L} = \left\{ z + ty \mid z \in L, t \in \mathbb{R}\right\}$.
            Очевидно, $\widehat{L}$ "--- линейное подпространство $Y$.
            Рассмотрим $\phi : X \rightarrow \mathbb{R}$, $\phi(z + ty) \defeq t$.
            По теорема Хана-Банаха его можно продлить на $Y$ с сохранением нормы:
            $\exists \widehat{\phi} \in Y^* : \restr{\widehat{\phi}}{\widehat{L}} = \phi$.
            Причем, если $z \in L$, то $\widehat{\phi}(z) = 0$, значит $\widehat{\phi} \in \Ker A^*$.
            Но, при этом $\widehat{\phi}(y) = 1 \implies y \not\in \left( \Ker A^*\right)^\bot$. \qedhere
    \end{enumerate}
\end{proof}
\begin{theorem}
    $R(A) = \Cl R(A) \implies R(A^*) = \left( \Ker A\right)^\bot$.
\end{theorem}
\begin{proof}
    Рассмотрим $f \in R(A^*)$. По определению, для некоторого $\phi$, $f = A^*\phi$.
    Возьмем теперь $x \in \Ker A$. $Ax = 0 \implies f(x) = (\phi \circ A)(x) = \phi(Ax) = \phi(0) = 0$.
    Значит, $R(A^*) \subset \left( \Ker A\right)^\bot$.
    
    Пусть теперь $f \in \left( \Ker A\right)^\bot$. В силу того, что $R(A)$ - $B$-пространство
    (как замкнутое линейное подпространство другого $B$-пространства),
    Возьмем произвольный $y \in R(A)$, и $x$ такой, что $y = Ax$,
    и запишем $\phi$ как $\phi(y) \defeq f(x)$. Покажем, что такое определение
    действительно корректное.
    Пусть $y = Ax'$; тогда $A(x - x') = 0 \implies x - x' \in \Ker A$.
    Поэтому $f(x - x') = 0 \implies f(x) = f(x')$. Это значит, что значение $\phi$ не
    зависит от выбора конкретного $x$. Значит, наша формула корректная.
    Осталось показать ограниченность $\norm{\phi}$.

    Рассмотрим ассоциированный оператор
    $\mathcal{U}_A : \sfrac{X}{\Ker A} \rightarrow R(A)$.
    Покажем, что он непрерывен.

    $\norm{\mathcal{U}_A} = \sup_{\norm{\eqcls{x}} = 1} \norm{\mathcal{U}_A\eqcls{x}}$,
    так как $\norm{\eqcls{x}} = \inf_{z \in \eqcls{x}} \norm{z} = 1$,
    то существует $x' \in \eqcls{x} : \norm{x'} \leqslant 2$.
    Возьмем $x'$ в качестве представителя. Тогда
    \begin{equation}
        \begin{split}
            \norm{\mathcal{U}_A} & =
            \sup_{\mathclap{\norm{\eqcls{x}} = 1}} \norm{\mathcal{U}_A\eqcls{x}}
            \leqslant \sup_{\mathclap{\norm{x} \leqslant 2}} \norm{Ax}
            \leqslant \sup_{\mathclap{\norm{y} \leqslant 1}} \norm{A(2y)} \\
            & = 2 \cdot \sup_{\mathclap{\norm{y} \leqslant 1}} \norm{Ay}
            = 2 \norm{A}
        \end{split}
    \end{equation}
    Заметим еще, что он биективен, так как все точки $x$ для которых $y = Ax$
    (для какого-то одного фиксированного $y$) лежат в одном классе эквивалентности.
    Это значит, что по теореме Банаха о гомеоморфизме, $\mathcal{U}_A^{-1}$ непрерывен.
    Напомним, что норма на элементах $\sfrac{X}{\Ker A}$ определяется как
    \begin{gather}
        \norm{\eqcls{x}} \defeq \inf_{\mathclap{z \in \eqcls{x}}} \norm{z}
    \end{gather}
    По непрерывности обратного оператора получаем $\norm{\eqcls{x}} \leqslant K \cdot \norm{y}$.
    Нам нужно сделать неравенство строгим, поэтому считаем, что
    $\norm{\eqcls{x}} < 2K \cdot \norm{y}$. Дальше, по определению инфимума,
    $\exists z \in \eqcls{x} : \norm{z} < 2K \cdot \norm{y}$. Значит, $z - x \in \Ker A$.
    В силу того, что значение функционала $f$ одно и то же внутри класса эквивалентности,
    можно вместо $x$ взять $z$. Таким образом,
    $\abs{\phi(y)} \leqslant 2K \cdot \norm{f} \cdot \norm{y}$, из этого следует, что
    $\phi$ непрерывен.
    Далее, по теореме Хана-Банаха, продолжим $\phi$ на все пространство и получим, что
    $\exists \widehat{\phi} \in Y^* : f = A^*\widehat{\phi} \implies f \in R(A^*)$.
\end{proof}

В силу того, что во второй теореме требуется замкнутость, 
возникает вопрос: а когда это действительно будет?
Одним из инструментов, дающих ответ на этот вопрос, является
\textit{априорная оценка решения операторного уравнения}.

\begin{definition}[априорная оценка решения операторного уравнения]
    Пусть $A : X \rightarrow Y$ - линейный оператор, 
    $y \in R(A)$, $\exists \alpha=\text{const}$, 
    такая что $\norm{x} \leqslant \alpha \norm{y}$,
    где $y = Ax$.
    Коэффициент $\alpha$ называется \textit{априорной оценкой}.
\end{definition}

Ответ на поставленный вопрос дает следующая теорема:
\begin{theorem}
    Если $A$ "--- линейный ограниченный оператор, такой что для уравнения 
    $y = Ax$ существует априорная оценка, то $R(A)$ "--- замкнуто.
\end{theorem}

\begin{proof}
    Рассмотрим последовательность значений оператора $y_n \in R(A)$, 
    такую что $y_n \rightarrow y$. Проверим, что тогда $y \in R(A)$.
    Пусть $\Epsilon_n = \frac{1}{2^n}$, в силу банаховости пространства $Y$, 
    можем написать ряд утверждений:
    \begin{gather*}
        \text{для } \Epsilon_1 \: \exists n_1 : \forall n, m \geqslant n_1
        \implies \norm{y_m - y_n} \leqslant \Epsilon_1 \\
        \text{для } \Epsilon_2 \: \exists n_2 : \forall n, m \geqslant n_2
        \implies \norm{y_m - y_n} \leqslant \Epsilon_2 \\
        \dots \\
        \text{для } \Epsilon_k \: \exists n_k : \forall n, m \geqslant n_k
        \implies \norm{y_m - y_n} \leqslant \Epsilon_k \\
        \dots \\
    \end{gather*}
    при этом, очевидно, что $n_k \leqslant n_{k + 1}$.
    Теперь рассмотрим ряд 
    $y_{n_1} + (y_{n_2} - y_{n_1}) + (y_{n_3} - y_{n_2}) + \dotso = y$.
    Слагаемое этого ряда мажорируется сходящейся геометрической прогрессией,
    поэтому, он сходится абсолютно. 
    
    Так как $R(A)$ "--- подпространство, значит $y_{n_{k + 1}} - y_{n_k} \in R(A)$.
    Следовательно, $y_{n_{k + 1}} - y_{n_k} = Ax_k$. По условию теоремы, для $x_k$
    выполняется $\norm{x_k} \leqslant \alpha \norm{y_{n_{k + 1}} - y_{n_k}} 
    \leqslant \alpha \Epsilon_k$. Возьмем ряд $x_0 + x_1 + x_2 + \dots$,
    где $y_{n_1} = Ax_0$. 
    Ряд из норм его слагаемых можно ограничить сходящимся рядом:
    $\norm{x_0} + \norm{x_1} + \norm{x_2} + \dotso \leqslant 
     \norm{x_0} + \alpha \cdot (\Epsilon_1 + \Epsilon_2 + \dotso) = 
     \norm{x_0} + \alpha$.
    Поэтому у него есть предел $x$, и мы можем применить к нему оператор почленно 
    (в силу его непрерывности):
    $Ax = Ax_0 + Ax_1 + Ax_2 + \dotso = y_{n_1} + (y_{n_{2}} - y_{n_1}) + 
    (y_{n_{3}} - y_{n_2}) + \dotso = y$. Таким образом, $y \in R(A)$.
\end{proof}

\pagebreak

\section{Элементы спектральной теории линейных операторов}

\subsection{Определение спектра и резольвенты оператора}

\begin{definition}[регулярная точка]
    Число $\lambda \in \mathbb{C}$, называется \emph{регулярной точкой}
    для оператора $A$, если оператор $\lambda I - A$ "--- непрерывно обратим.
\end{definition}

\begin{definition}[резольвента]
    Множество всех регулярных точек называется \emph{резольвентой} (обозначается $\rho(A)$)
    оператора $A$.
\end{definition}

\begin{definition}[резольвентный оператор]
    Оператор $R_\lambda(A) = \left( \lambda I - A\right)^{-1}$ называется
    \emph{резольвентным оператором}.
\end{definition}

\begin{definition}[спектр]
    Множество $\sigma(A) = \mathbb{C} \setminus \rho(A)$ называется
    \emph{спектром} оператора $A$.
\end{definition}

Рассмотрим $\lambda \in \sigma(A)$. Может быть два случая:
\begin{enumerate}
    \item
    $\Ker (\lambda I - A) \neq \left\{ \bf{0}\right\}$. Это значит, что оператор
    $\lambda I - A$ имеет нетривиальное собственное подпространство, в котором (по определению)
    выполняется $Ax=\lambda x$, $x \neq \bf{0}$, для некоторых $x$
    (то, что часто называется собственными числами и векторами).
    \item
    $\Ker (\lambda I - A) = \left\{ \bf{0}\right\}$.
    Здесь необходимо рассмотреть два подслучая:
    \begin{enumerate}
        \item
        $\dim X < +\infty$. В конечномерном случае из сюрьективности следует
        биективность, поэтому обратный оператор всегда существует.
        А спектр будет состоять из собственных значений.
        \item
        $\dim X = +\infty$. В этом случае может отсутствовать непрерывная обратимость.
        Если при этом $\Cl R(\lambda I - A) = X$, то говорят, что $\lambda$
        принадлежит непрерывной части спектра. Иначе говорят, что
        $\lambda$ принадлежит остаточной части спектра.
        (те $\lambda$ для которых ядро нетривиально называют дискретной частью спектра).
    \end{enumerate}
\end{enumerate}

\begin{statement}
    Резольвентное множество является открытым в $\mathbb{C}$.
\end{statement}
\begin{proof}
    Пусть $\lambda_0 \in \rho(A)$, тогда $\lambda_0 I - A$ - непрерывно обратим.
    Напишем тождество: $\lambda I - A = (\lambda - \lambda_0)I + \lambda_0I - A$.
    $I = (\lambda_0I - A) R_{\lambda_0}(A)$.
    Отсюда, $\lambda I - A = (\lambda_0I - A) \cdot
    (R_{\lambda_0} (\lambda - \lambda_0) - I)$.
    Заметим, что если $\norm{R_{\lambda_0}} \cdot
    \abs{\lambda - \lambda_0} < 1$,
    то по теореме Банаха о
    непрерывной обратимости оператора $I - C$,
    оператор $R_{\lambda_0} (\lambda - \lambda_0) - I$ - непрерывно обратим.
    Получается, что если $\lambda : \abs{\lambda - \lambda_0} <
    \frac{1}{\norm{R_{\lambda_0}}}$, то $\lambda \in \rho(A)$.
\end{proof}

\begin{corollary}
    Спектр - замкнутое множество.
\end{corollary}

\begin{theorem}
    Пусть $A$ - ограничен, тогда $\sigma(A) \neq \varnothing$
\end{theorem}
\begin{definition}[Спектральный радиус оператора]
    \begin{gather*}
        r_{\gamma}(A) \defeq \sup_{\lambda \in \sigma(A)} \abs{\lambda}
    \end{gather*}
\end{definition}
\begin{statement}
    $\exists \lim \sqrt[n]{\norm{A^n}} = r_{\gamma}(A)$
\end{statement}
\begin{proof}
    Очевидно, что всегда существует $\inf_{n}\sqrt[n]{\norm{A^n}} = r$.
    Итак, $\forall \epsilon > 0 \exists n_0: \sqrt[n_0]{\norm{A^{n_0}}} < r + \epsilon$.
    Рассмотрим теперь $n > n_0$. $n = m_n \cdot n_0 + d_n$. Тогда
    \begin{gather*}
        \norm{A^{n}}^{\frac{1}{n}}
        = \norm{A^{m_n \cdot n_0} \cdot A^{d_n}}^{\frac{1}{n}} \leqslant \norm{A^{n_0}}^{\frac{m_n}{n}} \cdot
        \norm{A}^{\frac{d_n}{n}}\text{, где } \norm{A}^{\frac{d_n}{n}} \rightarrow 0. \\
        \norm{A^{n_0}}^{\frac{m_n}{n}} < \left( \norm{A^{n_0}}^{\frac{1}{n_0}}\right)^{\frac{n_0 \cdot m_n}{n}}
        < \left( r + \epsilon\right)^{\frac{m_n \cdot n_0}{n}}
    \end{gather*}
    Следовательно, $r \leqslant \norm{A^n}^{\frac{1}{n}} \leqslant
    (1 + \alpha_n) \cdot \left( r + \epsilon\right)^{1 - \frac{d_n}{n}} =
    (1 - \alpha_n) \cdot (r - \epsilon)^{\frac{d_n}{n}} \cdot (r + \epsilon) =
    (1 + \gamma_n) \cdot (r + \epsilon)$.
\end{proof}

\begin{example}
    Пространство $C[0, 1]$. Оператор $A(f, t) = t \cdot f(t)$. Очевидно, что 
    $\norm{A(f)} \leqslant \norm{f}$. Пусть $\lambda I - A = A_\lambda$.
    $A_\lambda(f, t) = (\lambda - t) \cdot  f(t) = g(t) \implies 
    f(t) = \frac{g(t)}{\lambda - t}$. При каких $\lambda$, $A_\lambda$ непрерывно обратим?
    Очевидно, при $\lambda \notin [0, 1]$. Это значит, что если $\lambda \in [0, 1]  \implies \lambda \in \rho(A)$.
    Поэтому $\sigma(A) = [0, 1]$.
\end{example}

\begin{example}
    Пространство $C[0, 1]$. $A(f, x) = \int_{0}^{x}f(t)dt$, $x \in [0, 1]$.
    Вычислим его спектральный радиус.
    
    \begin{gather*}
        A(f, x) = \int_{0}^{x}f(t)dt \\
        A^2(f, x) = \int_{0}^{x} \left( \int_{0}^{x_1} f(t) dt\right)dx_1 \\
        A^n(f, x) = \int_{0}^{x} dx_1 \int_{0}^{x_1} dx_2 \cdots \int_{0}^{x_{n + 1}} f(t)dt \leqslant 
        \frac{\norm{f}}{n!} \\
        \norm{f} \leqslant 1 \implies \norm{A^n} \leqslant \frac{1}{n!} \implies r_n \leqslant \frac{1}{\sqrt[n]{n!}} \rightarrow 0 
        \implies r_{\sigma}(A) = 0
        \end{gather*}
\end{example}


\begin{theorem}[Об отображении спектра полиномами]
$\sigma(P(A)) = P(\sigma(A))$
\end{theorem}

\begin{lemma}
    
$P(A)$ - непрерывно обратим $\Leftrightarrow  0 \notin P(\sigma(A))$
\end{lemma}
\begin{proof}
    \begin{enumerate}
    \item Необходимость.

    Сначала проверим следующий факт. Если два оператора коммутируют и их произведение непрерывно обратимо, то и каждый из них непрерывно обратим.
    $T = A \cdot B = B \cdot A$, $\exists T^{-1}$ - непрерывно обратимый.
    $I = T^{-1} T = T^{-1}\cdot \left( A  B\right) = \left( T^{-1} A \right) \cdot B = B^{-1} B$. Для $A$ аналогично.
    В общем случае, $p(t)$ имеет вид $p(t) = a (t - \lambda_1)^{m_1} \cdot (t - \lambda_2)^{m_2} \cdots (t - \lambda_n)^{m_n}$, 
    а $p(A) = a (A - \lambda_1 I)^{m_1} \cdot (A - \lambda_2 I)^{m_2} \cdots (A - \lambda_n I)^{m_{n}}$. Так как $p(A)$ - непрерывно обратим, то, по доказанному,  
    каждый из множителей непрерывно обратим. Это значит, что каждое из $\lambda_j \in \rho(A)$. Поэтому, если $0 \in p(\sigma(A))$, 
    то одно из $\lambda_j$ является корнем уравнения $p(t) = 0$. То есть, при каком-то $\lambda_j$, $p(\lambda_j) = 0$, это значит что 
    $\lambda_j \in \sigma(A)$. Но $\lambda_j \in \rho(A)$. Противоречие.
    \item Достаточность. Тут все аналогично необходимости, только в другую сторону.
    

    \end{enumerate}
\end{proof}
\begin{proof}[Доказательство теоремы]
    Рассмотрим полином $p_1(t) = p(t) - \lambda$. Тогда, по лемме, $p_1(A)$ - непрерывно обратим тогда и только тогда когда $0 \notin p_1(\sigma(A))$, 
    что эквивалентно $p(A) - \lambda I$ - непрерывно обратим, тогда и только тогда когда $\lambda \notin p(\sigma(A))$. Но $\exists(p(A) - \lambda I)^{-1} \Leftrightarrow 
    \lambda \notin \sigma(p(A))$
\end{proof}

\subsection{Альтернатива Фредгольма-Шаудера}

\begin{definition}[Компактный оператор]
    Оператор $A: X \rightarrow Y$ называется \textit{компактным} если $\forall M$ - ограниченное множество, $A(M)$ - относительный компакт.
\end{definition}

\todo пример оператора фредгольма.

\begin{statement}[Компактность произведения]
    Пусть $A$ - компактный оператор, а $B$ - ограниченный. Тогда $AB$ и $BA$ - компактны.
\end{statement}
 
\begin{proof}
    Достаточно проверить компактность в единичном шаре.
    Пусть $V_1$ - замкнутый шар единичного радиуса, проверим, что $B(A(V_1))$ - относительный компакт.
    Так как $A(V_1)$ - относительный компакт, можем подобрать конечную $\epsilon$-сеть. 
    $\forall \epsilon \exists y_1,  \dots, y_n \forall x \in V_1 \exists j \norm{A(x) - y_j} < \epsilon$. Пусть $z_j = B(y_j)$. 
    $\forall x \in V_1 \norm{B(A(x)) - z_j} = \norm{B(A(x) - y_j)} \leqslant \norm{A(x) - y_j} \cdot \norm{B} \leqslant \epsilon\norm{B}$
\end{proof}

\begin{statement}
    В бесконечномерных пространствах, компактный оператор не может быть непрерывно обратимым.
\end{statement}
\begin{proof}
    Пусть $A$ - компактный, $\exists A^{-1}$ - непрерывный. Тогда $I = A\cdot A^{-1}$ - компактный. Однако, это не так (потому что бесконечномерная сфера - не компакт).
\end{proof}

В классе сепарабельных банаховых пространств важную роль имеют пространства с базисом \textit{Шаудера}.

\begin{definition}[базис Шаудера]
    Пусть $X$ - баназово пространство. $\exists e_1, \dots, e_n, \dots$ - линейно независимые точки. $\forall x \in X$, $x = \sum_{1}^{+\infty} \alpha_j e_j$.
    Тогда ${e_j}$ - называется \textit{базисом Шаудера}.
\end{definition}

\begin{theorem}[О почти конечномерности компактного оператора]
    Пусть $A: X \rightarrow X$ - компактный оператор. $X$ - имеет базис Шаудера. Тогда 
    $\forall \epsilon > 0 \exists B, C: \dim R(B) < +\infty, \norm{C} \leqslant \epsilon, A = B + C$
\end{theorem}


\section{Теорема Гильберта-Шмидидта}
\begin{definition}[Гильбертово пространство]
    $H$ - гильбертово пространство над $\mathbb{C}$, если.
    \begin{itemize}
        \item $\scalarp{x}{y} = \overline{\scalarp{y}{x}}$
        \item $\scalarp{x}{\alpha y} = \bar{\alpha}\scalarp{x}{y}$
    \end{itemize}
\end{definition}

\begin{example}[$\mathbb{C}^n$ -- конечномерный случай]
    $z = (z_1, z_2, \dotsc, z_n), z_i \in \mathbb{C} \\
    \scalarp{z}{u} = \sum_{i = 1} ^n z_i \bar{u_i} \\
    \scalarp{x}{x} \geqslant 0, \scalarp{x}{x} = 0 \Leftrightarrow x = 0 \\
    \scalarp{x}{x} = \sum_{i = 1} ^n x_i \cdot \bar{x_i} = \sum_{i = 1} ^n |x_i|^2$
\end{example}

\begin{definition}[Самосопряженный оператор]
    Линейный оператор $A: X \rightarrow X$ называется самосопряженным, 
    если выполняется условие:
    $\scalarp{Ax}{y} = \scalarp{x}{Ay}$.
\end{definition}

\begin{example}
    $A : \mathbb{C}^n \rightarrow \mathbb{C}^n$, 
    $A = \bar{A}^T$ -- эрмитовски симметричная
\end{example}

\begin{statement}
    Если  $\int_a ^b K(k, t) f(t) dt$ - интегральный оператор, 
    то $K(x, y) = \overline{K(y, x)}$
\end{statement}
\begin{proof}
    \todo{\newline Домножить векторы на матрицу поворота}
\end{proof}

\begin{statement}
    $A$ -- самосопряженный оператор, $\lambda = \mu + i \nu$.
    Тогда $\forall x \in H \Rightarrow 
    \norm{Ax - \lambda x} \geqslant |\nu| \cdot \norm{x}$
\end{statement}
\begin{proof}\
    \begin{enumerate}
    	\item Если матрица состоит из вещественных чисел:
            $\norm{Ax - \lambda x}^2 = \scalarp{Ax - \lambda x}{Ax - \lambda x} = \\ 
            = (\underbrace{Ax - \mu x} _{Lx, \mu \in \mathbb{R}} - i \nu x,
            Ax - \mu x - i \nu x) =
            \underbrace{\scalarp{Lx}{Lx}}_{\norm{Lx}^2 = 0} - 
            \underbrace{
                \underbrace{\scalarp{Lx}{i \nu x}} _{- i \nu \scalarp{Lx}{x}} - 
                \underbrace{\scalarp{i \nu x}{Lx}}_{i \nu \scalarp{Lx}{x}}
            }_{= 0} + 
            |\nu|^2 \norm{x}^2 \geqslant |\nu|^2 \norm{x}^2$
        \item Если $Im(a) \neq 0, \\ 
            Ker(A - \lambda i) = \{0\} \Rightarrow$
            Cобственные числа могут быть только вещественными.
            $\Rightarrow |\nu| = |\nu|\norm{x} = 0$
    \end{enumerate}
\end{proof}

\begin{statement}
    Пусть $A$ - линейный оператор в $H$, тогда $Cl\, R(A) = (Ker A^*)^{\perp}$
\end{statement}
\begin{proof}
    \begin{nb}
        В общем случае уже доказывали, сейчас выведем независимо.


        \todo{Написать, где доказательство.}
    \end{nb}
    \begin{enumerate}
        \item Рассмотрим $y \in R(A): \\
            \forall z \in Ker A^* \Leftrightarrow
            A^*z = 0 \Rightarrow
            \scalarp{y}{z} = \scalarp{Ax}{z} = \scalarp{x, A^*z} = \scalarp{x}{0} = 0 \\
            \text{Taким образом, } \forall y \in R(A) \Rightarrow
            y \perp A^* \\
            y \in Cl\, R(A), y_n \in R(A), y_n \rightarrow y \Rightarrow
            0 = \scalarp{y_n}{z} \rightarrow \scalarp{y}{z} \Rightarrow
            \scalarp{y}{z} = 0 \\
            $Итак, $ Cl\, R(A) \subset (Ker A^*)^{\perp}.$
        \item Проверим обратное включение
            $y \notin Cl\, R(A) \Rightarrow
            y \notin (Ker A^*)^{\perp}. \\
            Cl\, R(A) = H_1$ -- подпространство $H,
            H = H_1 \oplus H_1 ^\perp. \\
            $Рассмотрим $ y \notin H_1, 
            \exists y_1 \in H_1, y_2 \in H_1 ^\perp, y_2 \neq 0 :\,
            y = y_1 + y_2 \\
            y_2 \in Ker A^* \Leftrightarrow
            A^*y_2 = 0 \Leftrightarrow
            \scalarp{A^*y_2}{A^*y_2} = 0 \Leftrightarrow
            (\underbrace{A(A^*y_2)}_{\in H_1},
            \underbrace{y_2}_{\in H_1 ^\perp}) = 0 \\
            \scalarp{y}{y_2} = \norm{y_2}^2 \neq 0 \Rightarrow
            y \notin (KerA^*)^\perp$
    \end{enumerate}
\end{proof}
\begin{corollary}
    $H = Cl\, R(A) \oplus Ker A^*$
\end{corollary}

\begin{theorem}
    Пусть оператор $A$ самосопряженный,тогда $\sigma(A) \subset R$.
\end{theorem}
\begin{proof}
    $\lambda : Im \lambda \neq 0,\,
    R(A - \lambda x)$ -- замкнуто $\\
    (A - \lambda I)^* = A - \bar{\lambda} I \\
    \norm{(A -\bar{\lambda} I)x} \geqslant
    \underbrace{|Im \lambda|}_{\neq 0} \cdot \norm{x} \\
    Ker(A - \lambda I)^* = \{0\}, H = R(A - \lambda I) \\
    По теореме Банаха о гомоморфизме: \exists (A - \lambda I)^{-1}
    \Rightarrow \lambda \in \dot{\rho}(A)$
\end{proof}

\begin{theorem}
    $A$ -- самосопряженный и ограниченный в $H$.
    \begin{enumerate}
        \item
            $\lambda \in \rho \Leftrightarrow 
            \exists m > 0: \forall x \in H \quad
            \norm{(A - \lambda I)x)} \geqslant m \norm{x}$
        \item
            $\lambda \in \sigma(A) \Leftrightarrow
            \exists x_m: \norm{x_m} = 1 \quad
            \norm{(A - \lambda I)x_m} \rightarrow 0$
    \end{enumerate}
\end{theorem}
\begin{proof}\
    \begin{enumerate}
        \item $\Rightarrow: \\
            \lambda \in \rho (A) \Leftrightarrow \exists (A - \lambda I)^{-1}
            \norm{(A - \lambda I)^{-1}} \cdot M \\
            \norm{(A - \lambda I)^{-1}y} \leqslant M \norm{y} \\
            y = (A - \lambda I)x \\
            \frac{m}{M} \norm{x} \leqslant \norm{(A -\lambda I)x} \\
            \\ \Leftarrow: \\$
            Если неравенство выполнено, то $R(A - \lambda I)$ - замкнут.$\\
            Im(\lambda) \neq 0,\, \lambda = \bar{\lambda},
            (A - \lambda I)^* = A - \lambda I,\,
            Ker(A - \lambda I) = \{0\} \\
            H = R(A - \lambda I)\,$
            Значит, он сюрьективный.
        \item
            Математическое отрицание пункта 1,
            но $Ax_n \approx \lambda x_n. \\$
            То есть, у самосопряженного оператора спектр состоит из почти собственных значений.
            \begin{gather}
                \scalarp{Ax}{x} = \overline{\scalarp{x}{Ax}} =
                \overline {\scalarp{Ax}{x}} \Rightarrow
                \forall x \in H \quad \scalarp{Ax}{x} \in \mathbb{R}
            \end{gather}
            Это называется кр. ф-ой самосопряженного оператора.

            \todo{ Расшифровать сокращения в предыдущей строчке.}
    \end{enumerate}
\end{proof}

\begin{definition}
    Нижняя и верхняя граница оператора $A \\
    m_- = \inf_{\norm{x} = 1} \scalarp{Ax}{x} \quad
    m_+ = \sup_{\norm{x} = 1} \scalarp{Ax}{x}$
\end{definition}

\begin{statement}
    $|\scalarp{Ax}{x}| \leqslant \norm{Ax} \cdot \norm{x} \leqslant
    \norm{A} \cdot \norm{x}^2 = \norm{A} \\$
    То есть $|m_-|, |m_+| \leqslant$
\end{statement}
\begin{proof}\
    $m_- \leqslant \scalarp{Ax}{x} \leqslant m_+ \quad \norm{x} = 1 \\
    m_- \underbrace{\norm{x}^2}_{\scalarp{x}{x}}
    \leqslant \scalarp{Ax}{x} \leqslant m_+ \norm{x}^2 \Rightarrow
    0 \leqslant \scalarp{A - m_- I)x}{x}$
\end{proof}

\begin{theorem}\
    \begin{enumerate}
    	\item $\sigma(A) \subset [m_-, m_+]$
    	\item $m_-, m_+ \in \sigma(A)$
    \end{enumerate}
\end{theorem}
\begin{proof}\
    \begin{enumerate}
    	\item
    	    Так как числа с ненулевой мнимой частью резольв. множество,
    	    рассмотрим только вещественные.
    \end{enumerate}
    \todo{ Расшифровать сокращение "резольв".}
\end{proof}
\pagebreak
\end{document}
